\documentclass{article}

\usepackage{amsthm}
\usepackage{graphicx}
\graphicspath{ {./images/} }

\newtheoremstyle{mydef}
{\topsep}{\topsep}%
{}{}%
{\bfseries}{}
{\newline}
{%
  \rule{\textwidth}{0.4pt}\\*%
  \thmname{#1}~\thmnumber{#2}\thmnote{\ -\ #3}.\\*[-1.5ex]%
  \rule{\textwidth}{0.4pt}}%

\theoremstyle{mydef}
\newtheorem{definition}{Definition}

\usepackage[hybrid]{markdown}


\title{Notes to Google's Machine Learning Crash Course}
\author{Michael Sendker}
\date{June 27, 2021}

\begin{document}
\begin{markdown}
\maketitle

# Definitions

\begin{definition}[label]
    A label is Y
\end{definition}

\begin{definition}[feature]
    The feature is X
\end{definition}

\begin{definition}[inference]
    running trained model on unlabeled data
\end{definition}

\begin{definition}[regression]
    predicting continuous values (e.g., median house values)
\end{definition}

\begin{definition}[classification]
    predicting discrete values (e.g., hot dog, not a hot dog) also includes
    multi-value classification (e.g., dog, cat, or hamster)
\end{definition}

\begin{definition}[hyperparameters]
    the configuration settings used to tune how the model is trained
\end{definition}

\begin{definition}[empirical risk minimization]
    In supervised learning, a machine learning algorithm builds a model by
    examining many examples and attempting to find a model that minimizes loss;
    this process is called **empirical risk minimization**.
\end{definition}

# Conceptual understanding

Tom helped me understand: all ML is is finding a plane in n-dimensional space
that segregates distinct Ys. The different methods, like Forests, Trees, etc.,
are just different methods of generating functions to distinguish those Ys from
their Xs.

# Reducing Loss

\begin{equation}
    L_2Loss = \sum{(x,y) \in D} (y - prediction(x))^2
\end{equation}

Sometimes useful to average over all examples so divide by $\|D\|$

We use the square of the difference to get a nice concave graph that allows easy
stepping to reduce the loss.


The Derivative of (y-y')² with respect to the weights and biases tells us how
loss changes for a given example.

Repeatedly take small steps in the direction that minimizes loss. These are
called (negative) Gradient Steps. This strategy is called Gradient Descent

\begin{definition}[Gradient Descent]
    The strategy whereby one takes repeated small steps in the direction that
    minimizes loss.
\end{defintion}\\\\

If your batch is large, testing a lot of steps is extremely expensive, you could
test with one example at a time, or do batches of 10-1000 and average over time.
One example at a time is called **Stocahastic Gradient Descent** and small
batches is **Mini-Batch Gradient Descent**.

I really want to understand the mathy stuff because it's super cool and hard and
I believe I can do it, so, infobox from Google:

# Partial Derivatives

A multivariable function is a function with more than one argument, such as:

\begin{equation}
    f(x,y) = e^2y \sin{x}
\end{equation*}

The partial derivative of $f$ with respect to $x$, denoted as

\begin{equation}
    \frac{\partial f}{\partial x}
\end{equation*}

is the derivative of $f$ as a function of $x$ considered alone. To find
$\frac{\partial f}{\partial x}$ hold $y$ constant and take the derivative of $f$
with respect to $x$.

That seems easy enough. I'm copying a lot of this verbatim.

In general, thinking of $y$ as fixed, the partial derivative of $f$ with respect
to $x$ is calculated as follows:

\begin{equation}
    \frac{\partial f}{\partial x}(x,y) = e^2y\cos{x}
\end{equation*}

\end{markdown}

\end{document}
